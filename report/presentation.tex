\documentclass[9pt]{beamer}
% Created By Gouthaman KG
%~~~~~~~~~~~~~~~~~~~~~~~~~~~~~~~~~~~~~~~~~~~~~~~~~~~~~~~~~~~~~~~~~~~~~~~~~~~~~~
% Use roboto Font (recommended)
\usepackage[sfdefault]{roboto}
\usepackage[utf8]{inputenc}
\usepackage[T1]{fontenc}
%~~~~~~~~~~~~~~~~~~~~~~~~~~~~~~~~~~~~~~~~~~~~~~~~~~~~~~~~~~~~~~~~~~~~~~~~~~~~~~

%~~~~~~~~~~~~~~~~~~~~~~~~~~~~~~~~~~~~~~~~~~~~~~~~~~~~~~~~~~~~~~~~~~~~~~~~~~~~~~
% Define where theme files are located. ('/styles')
\usepackage{styles/fluxmacros}
\usefolder{styles}
% Use Flux theme v0.1 beta
% Available style: asphalt, blue, red, green, gray
\usetheme[style=asphalt]{flux}
%~~~~~~~~~~~~~~~~~~~~~~~~~~~~~~~~~~~~~~~~~~~~~~~~~~~~~~~~~~~~~~~~~~~~~~~~~~~~~~

%~~~~~~~~~~~~~~~~~~~~~~~~~~~~~~~~~~~~~~~~~~~~~~~~~~~~~~~~~~~~~~~~~~~~~~~~~~~~~~
% Extra packages for the demo:
\usepackage{booktabs}
\usepackage{multirow}
\usepackage{colortbl}
\usepackage{ragged2e}
\usepackage{schemabloc}
\usepackage{hyperref}
\hypersetup{
    colorlinks=true,
    urlcolor=purple,
    linkcolor=.
}

\usepackage{caption,subcaption}
\usebackgroundtemplate{
\includegraphics[width=\paperwidth,height=\paperheight]{images/background.jpg}}
\setbeamertemplate{caption}[numbered]

% Informations
\title{Project presentation}
\subtitle{Applying tabu search to the capacitated vehicle routing problem}

\author{Phan Ngoc Lan}
\institute{Hanoi University of Science and Technology}
\titlegraphic{images/hust.png} %change this to your preferred logo or image(the image is located on the top right corner).
%~~~~~~~~~~~~~~~~~~~~~~~~~~~~~~~~~~~~~~~~~~~~~~~~~~~~~~~~~~~~~~~~~~~~~~~~~~~~~~

\begin{document}
\AtBeginSection[]{
    \begin{frame}<beamer>
        \frametitle{Outline}
        \tableofcontents[currentsection]
    \end{frame}
}

% Generate title page
\titlepage

\begin{frame}
 \frametitle{Table of contents}
 \tableofcontents
\end{frame}

\section{The capacitated vehicle routing problem}
\begin{frame}{Vehicle routing problems}
\begin{block}{}
    What is the optimal set of routes for a fleet of vehicles to traverse in order to deliver to a given set of customers?
\end{block}

Proposed by Dantzig and Ramser, vehicle routing problems (VRP) occur naturally in multiple fields, most notably transportation and logistics.

There are many variants of the VRP:
\begin{itemize}
    \item Capacitated Vehicle Routing Problem (CVRP)
    \item Vehicle Routing Problem with Time Windows (VRPTW)
    \item Vehicle Routing Problem with Profits (VRPP)
    \item etc,\dots
\end{itemize}
\end{frame}

\begin{frame}{Capacitated vehicle routing}
The capacitated vehicle routing problem (CVRP) is one of the most basic and well-studied variant of VRP.

Each customer has a demand index that needs to be fulfilled. Each vehicle has a capacity for service.
\end{frame}

\begin{frame}{Mathematical formulation}
The mathematical formulation for CVRP is modified from TSP:

\begin{equation}
    min \sum_{i \in V} \sum_{j \in V} c_{ij}x_{ij}
\end{equation}
subject to
\begin{align}
    \sum_{i \in V} x_{ij} & = 1 \ \forall j \in V \backslash \{0\} \\
    \sum_{j \in V} x_{ij} & = 1 \ \forall i \in V \backslash \{0\} \\
    \sum_{i \in V} x_{i0} & = K \\
    \sum_{j \in V} x_{0j} & = K \\
    \sum_{i \notin S} \sum_{j \in S} x_{ij} & \geq r(S), \forall S \subseteq V \backslash \{0\},\ S \neq \emptyset\\
    x_{ij} & \in \{ 0, 1 \} \forall i,j \in V
\end{align}

\end{frame}

\section{Related works}
\begin{frame}{Related works}

\end{frame}

\section{Tabu search}
\begin{frame}{Overview}
Tabu search is a local search paradigm that uses memory structures to prevent the search from visiting previous solutions.

Proposed by Glover and Hansen in 1986.

Full tabu search includes 3 phases:
\begin{itemize}
    \item Short-term: Search using tabu list starting from the initial solution
    \item Intensification: A good solution is selected and the search focuses on this solution's neighborhood
    \item Diversification: The search explores a different area in the search space to find better solutions
\end{itemize}
\end{frame}

\begin{frame}{SimpleTabu for CVRP}

\end{frame}

\begin{frame}{SimpleTabu - Initialization}
    The initial solution is generated using the Clarke-Wright greedy algorithm (savings algorithm). Each edge $(i, j)$ is assigned a savings value, which is the amount of cost that can be saved by joining the routes $0, i, 0$ and $0, j, 0$. Edges are then greedily joined from the highest weighted edge to the lowest.

    The savings algorithm is deterministic. To add randomness, we can shuffle the savings list or assign random weight to each edge's savings value.

    SimpleTabu initializes a solution by randomizing the Clarke-Wright algorithm.
\end{frame}

\begin{frame}{SimpleTabu - Exploration}

\end{frame}

\begin{frame}{SimpleTabu - Intensification}

\end{frame}

\begin{frame}{SimpleTabu - Diversification}

\end{frame}

\section{Experiments}
\begin{frame}{Experiment settings}
2 experiments are performed:
\begin{enumerate}
    \item Ablation study: 4 variations of SimpleTabu with different operators are compared
    \item Comparison: SimpleTabu is compared with ALNS (large neighborhood search) and GRELS (genetic algorithm + local search)
\end{enumerate}

All experiments use the Christofides et al. 1979 dataset. Includes 14 instances of medium size.

Results for \href{http://www.vrp-rep.org/references/item/pisinger-and-ropke-2007.html}{ALNS} and \href{http://www.vrp-rep.org/references/item/prins-2009.html}{GRELS} are reported at VRP-REP.

SimpleTabu is implemented in Python3. Experiments are run using the PyPy runtime on a 2-core AMD EPYC 2.2Ghz CPU with 4GB RAM.
\end{frame}

\begin{frame}{Ablation study - Results}
\begin{table}[]
\resizebox{\textwidth}{!}{
\begin{tabular}{ |c|r|r|r|r|r|r|r|r| }
\hline
\multicolumn{1}{|c|}{ \multirow{2}{*}{ \textbf{ Instance } } } & \multicolumn{2}{|c|}{ \textbf{ SimpleTabu (Full) } } & \multicolumn{2}{|c|}{ \textbf{ SimpleTabu (No Or-Opt) } } & \multicolumn{2}{|c|}{ \textbf{ SimpleTabu (No 2-Opt*) } } & \multicolumn{2}{|c|}{ \textbf{ SimpleTabu (No Relocate) } } \\
\cline{ 2-9 }
 & \multicolumn{1}{|c|}{ Best } & \multicolumn{1}{|c|}{ Std } & \multicolumn{1}{|c|}{ Best } & \multicolumn{1}{|c|}{ Std } & \multicolumn{1}{|c|}{ Best } & \multicolumn{1}{|c|}{ Std } & \multicolumn{1}{|c|}{ Best } & \multicolumn{1}{|c|}{ Std } \\ \hline
CMT01 &  581.02 & 10.84 & \textbf{ 570.24 } & 17.04 &  593.18 & \textbf{ 8.23 } &  595.83 & 53.00 \\ \hline
CMT02 & \textbf{ 900.68 } & 17.59 &  911.27 & 16.73 &  934.70 & \textbf{ 12.35 } &  960.40 & 32.64 \\ \hline
CMT03 &  918.25 & \textbf{ 7.61 } & \textbf{ 905.49 } & 29.29 &  912.26 & 18.47 &  952.33 & 27.77 \\ \hline
CMT04 &  1186.45 & 29.41 &  1215.15 & \textbf{ 15.07 } & \textbf{ 1179.75 } & 27.01 &  1228.61 & 63.68 \\ \hline
CMT05 & \textbf{ 1450.05 } & 32.45 &  1492.80 & \textbf{ 28.91 } &  1494.17 & 37.18 &  1531.85 & 82.98 \\ \hline
CMT06 & \textbf{ 556.60 } & \textbf{ 10.18 } &  556.73 & 16.25 &  584.93 & 26.39 &  584.58 & 52.33 \\ \hline
CMT07 &  903.96 & \textbf{ 6.83 } & \textbf{ 887.46 } & 17.72 &  906.42 & 24.12 &  947.14 & 31.14 \\ \hline
CMT08 & \textbf{ 890.38 } & 14.72 &  928.84 & 16.24 &  916.09 & \textbf{ 8.94 } &  936.41 & 52.14 \\ \hline
CMT09 & \textbf{ 1164.78 } & 20.38 &  1170.63 & 40.81 &  1224.42 & \textbf{ 14.77 } &  1235.59 & 55.95 \\ \hline
CMT10 & \textbf{ 1451.19 } & 41.34 &  1494.38 & 18.74 &  1522.03 & \textbf{ 9.84 } &  1564.61 & 73.23 \\ \hline
CMT11 &  1076.37 & 31.40 & \textbf{ 1076.30 } & \textbf{ 7.95 } &  1077.94 & 29.20 &  1203.77 & 32.97 \\ \hline
CMT12 & \textbf{ 835.33 } & 12.19 &  844.62 & \textbf{ 4.71 } &  841.37 & 6.51 &  863.25 & 34.39 \\ \hline
CMT13 & \textbf{ 1073.88 } & \textbf{ 6.31 } &  1085.91 & 34.68 &  1085.00 & 10.21 &  1312.98 & 40.21 \\ \hline
CMT14 & \textbf{ 829.80 } & \textbf{ 5.29 } &  835.56 & 6.98 &  835.06 & 6.28 &  923.15 & 9.77 \\ \hline
\end{tabular}
}
\caption{ Best total cost and standard deviation over 5 runs for variations of SimpleTabu on the Christofides 1979 dataset }
\label{ tab:compare-abl-cmt }
\end{table}

\end{frame}

\begin{frame}{Ablation study - Discussion}

\end{frame}

\begin{frame}{Comparison - Results}
\begin{table}[]
\centering
\begin{tabular}{ |c|r|r|r| }
\hline
\multicolumn{1}{|c|}{ \textbf{ Instance } } & \multicolumn{1}{c|}{ \textbf{ SimpleTabu } } & \multicolumn{1}{c|}{ \textbf{ ALNS } } & \multicolumn{1}{c|}{ \textbf{ GRELS } } \\ \hline
CMT01 & 581.02 & \textbf{ 524.61 } & \textbf{ 524.61 } \\ \hline
CMT02 & 900.68 & \textbf{ 835.26 } & \textbf{ 835.26 } \\ \hline
CMT03 & 918.25 & \textbf{ 826.14 } & \textbf{ 826.14 } \\ \hline
CMT04 & 1186.45 & 1029.56 & \textbf{ 1029.48 } \\ \hline
CMT05 & 1450.05 & 1297.12 & \textbf{ 1294.09 } \\ \hline
CMT06 & 556.60 & \textbf{ 555.43 } & \textbf{ 555.43 } \\ \hline
CMT07 & \textbf{ 903.96 } & 909.68 & 909.68 \\ \hline
CMT08 & 890.38 & \textbf{ 865.94 } & \textbf{ 865.94 } \\ \hline
CMT09 & 1164.78 & 1163.68 & \textbf{ 1162.55 } \\ \hline
CMT10 & 1451.19 & 1405.88 & \textbf{ 1401.46 } \\ \hline
CMT11 & 1076.37 & 1042.12 & \textbf{ 1042.11 } \\ \hline
CMT12 & 835.33 & \textbf{ 819.56 } & \textbf{ 819.56 } \\ \hline
CMT13 & \textbf{ 1073.88 } & 1542.86 & 1545.43 \\ \hline
CMT14 & \textbf{ 829.80 } & 866.37 & 866.37 \\ \hline
\end{tabular}
\caption{ Best total cost for SimpleTabu, ALNS and GRELS on the Christofides 1979 dataset }
\label{ tab:compare-others-cmt }
\end{table}

\end{frame}

\begin{frame}{Comparison - Discussion}

\end{frame}

\begin{frame}{Convergence analysis}
\begin{figure}[ht]
    \centering
    \begin{subfigure}[b]{0.48\linewidth}
        \centering
        \includegraphics[width=0.95\textwidth]{images/converge_1.jpg}
        \caption{CMT07}
    \end{subfigure}
    \begin{subfigure}[b]{0.48\linewidth}
        \centering
        \includegraphics[width=0.95\textwidth]{images/converge_2.jpg}
        \caption{CMT12}
    \end{subfigure}
    \caption{Sample convergence charts for SimpleTabu}
\end{figure}
\end{frame}

\section{Conclusion}
\begin{frame}{Conclusion}

\end{frame}

\end{document}
